\section*{Trigonometrie}

\subsection*{Sinus und Cosinus}
$\displaystyle \cos(x) := \sum_{n=0}^\infty (-1)^n\cdot\frac{x^{2n}}{(2n)!}$ ($r=\infty$) \\
$\displaystyle \sin(x) := \sum_{n=0}^\infty (-1)^n\cdot\frac{x^{2n+1}}{(2n+1)!}$ ($r=\infty$)

\subsubsection*{Im Komplexen}
$\cos(x)=\frac{1}{2}\cdot(e^{ix}+e^{-ix})$ \\
$\sin(x)=\frac{1}{2i}\cdot(e^{ix}-e^{-ix})$ \\
$\exp(ix)=\cos(x)+i\cdot\sin(x)$

\subsubsection*{Additionstheoreme}
$\cos(x+y)=\cos(x)\cdot\cos(y)-\sin(x)\cdot\sin(y)$\\
$\sin(x+y)=\cos(x)\cdot\sin(y)+\sin(x)\cdot\cos(y)$

\subsubsection*{Weiteres}
$\cos^2(x)+\sin^2(x)=1$

\subsubsection*{Wertetabelle}
\begin{tabular}{c|c|c|c|c|c}
	  & $0$ & $\frac{\pi}{6}$ & $\frac{\pi}{4}$ & $\frac{\pi}{3}$ & $\frac{\pi}{2}$ \\ \midrule
	$\sin$ & $0$ & $\frac{1}{2}$ & $\frac{1}{\sqrt{2}}$ & $\frac{\sqrt{3}}{2}$ & $1$ \\
	$\cos$ & 1 & $\frac{\sqrt{3}}{2}$ & $\frac{1}{\sqrt{2}}$ & $\frac{1}{2}$ & $0$ \\
\end{tabular}

\subsection*{Tangens und Cotangens}
$\tan := \frac{\sin}{\cos}$, $\cot := \frac{\cos}{\sin}$

\subsection*{Ableitungen}
$\sin'(x)=\cos(x)$, $\arcsin'(x)=\frac{1}{\sqrt{1-x^2}}$ \\
$\cos'(x)=-\sin(x)$, $\arccos'(x)=-\frac{1}{\sqrt{1-x^2}}$ \\
$\tan'(x)=\frac{1}{\cos^2(x)}$, $\arctan'(x)=\frac{1}{1+x^2}$ \\
$\cot'(x)=-\frac{1}{\sin^2(x)}$, $\arccot'(x)=-\frac{1}{1+x^2}$
