\documentclass[10pt,landscape,a4paper]{article}
\usepackage[ngerman]{babel}
\usepackage[T1]{fontenc}
\usepackage[dvipsnames]{xcolor}
%\usepackage[LY1,T1]{fontenc}
\usepackage{tikz}
\usetikzlibrary{shapes,positioning,arrows,fit,calc,graphs,graphs.standard}
\usepackage[nosf]{kpfonts}
\usepackage{multicol}
\usepackage{wrapfig}
\usepackage[top=2mm,bottom=3mm,left=2mm,right=2mm,headsep=1mm,includehead]{geometry}
\usepackage[framemethod=tikz]{mdframed}
\usepackage{microtype}
\usepackage{pdfpages}
\usepackage{csquotes}
\usepackage{booktabs}

\usepackage{amsmath}
\usepackage{bbm}
\newcommand{\N}{\mathbb{N}}
\newcommand{\Z}{\mathbb{Z}}
\newcommand{\R}{\mathbb{R}}
\newcommand{\C}{\mathbb{C}}
\newcommand{\F}{\mathbb{F}}
\newcommand{\K}{\mathbb{K}}
\renewcommand{\S}{\mathbb{S}}
\DeclareMathOperator{\im}{Im}
\DeclareMathOperator{\Ker}{Ker}
\DeclareMathOperator{\Sym}{Sym}
\DeclareMathOperator{\sgn}{sgn}
\DeclareMathOperator{\Hom}{Hom}
\DeclareMathOperator{\End}{End}
\DeclareMathOperator{\lc}{lc}
\DeclareMathOperator{\Mat}{Mat}
\DeclareMathOperator{\Gl}{Gl}
\DeclareMathOperator{\Lin}{Lin}
\DeclareMathOperator{\id}{id}
\DeclareMathOperator{\rang}{rang}
\DeclareMathOperator{\rZSF}{rZSF}
\DeclareMathOperator{\Spur}{Spur}
\DeclareMathOperator{\Eig}{Eig}

\newcommand\mat[1]{
	\begin{pmatrix}
		#1
	\end{pmatrix}
}

\usepackage{blindtext}

\usepackage{titlesec}
\newcommand{\sectionrule}{ \par \hrule \nopagebreak \bigskip }

\titleformat{\section}{\sectionrule\large\color{Mahogany}}{\thesection}{1em}{\scshape}
\titleformat{\subsection}{\color{RoyalPurple}}{\thesubsection}{1em}{}
\titleformat{\subsubsection}{}{\thesubsubsection}{1em}{\itshape}

\titlespacing*{\section}{0cm}{0.4\baselineskip}{0.4\baselineskip}
\titlespacing*{\subsection}{0cm}{0.3\baselineskip}{0.3\baselineskip}
\titlespacing*{\subsubsection}{0cm}{0.2\baselineskip}{0.2\baselineskip}


\usepackage{hyperref}
\newcommand\nohyper[1]{\begin{NoHyper}#1\end{NoHyper}}

\usepackage{lastpage,fancyhdr}
\pagestyle{fancy}
\fancyhf{}
\fancyhead[L]{Rasmus Buurman}
\fancyhead[C]{\textsc{Lineare Algebra I}}
\fancyhead[R]{Seite \thepage{}/\nohyper{\pageref{LastPage}}}
\renewcommand\headrulewidth{0.5pt}

\let\bar\overline

\begin{document}
\small
\begin{multicols*}{5}
	\section*{Gruppen \& Homomorphismen}

\subsection*{Gruppe $(G,\times)$}
\begin{itemize}
	\item $(g \times h) \times k = g \times (h \times k)$
	\item $\exists e \in G: \forall g\in G: e\times g = g$
	\item $\forall g\in G: \exists g^{-1}\in G: g^{-1} \times g = e$
\end{itemize}
In abelschen Gruppen: $g\times h = h \times g$

\subsubsection*{Untergruppe $U\le G$}
$\forall u,v\in U: u\times v \in U$ und $u^{-1}\in U$

\subsection*{Erzeugnis}
Gruppe $G$ und $M \subseteq G$.

$\displaystyle \langle M \rangle := \bigcap_{M\subseteq U \le G}U$ \\
$= \{g_1^{\alpha_1}\times\cdots\times g_n^{\alpha_n} \mid g_i\in M,\alpha_i\in\Z\} $

\subsection*{Gruppenhomomorphismus $\varphi:G \to H$}
$(G, \times)$ und $(H, *)$ Gruppen. \\
$\forall g,g'\in G: \varphi(g\times g')=\varphi(g) * \varphi(g')$

\subsubsection*{Bild und Kern}
\begin{itemize}
	\item $\im(\varphi):=\varphi(G) \le H$
	\item $\Ker(\varphi) := \varphi^{-1}(e_H) \le G$
	\item $\varphi$ injektiv $\iff \Ker(\varphi)=\{e_G\}$
\end{itemize}

\subsection*{Die symmetrische Gruppe $\S_n$}
$\S_n := \Sym(\{1,\dots,n\})$\\
$=\{\sigma \mid \sigma \text{ ist Permutation}\}$\\
($\sigma : \{1,\dots,n\}\to\{1,\dots,n\}$ bijektiv)

\subsubsection*{Zyklen und Transpositionen}
$\sigma = (a_1 \dots a_k)$ ist \textit{k-Zyklus}:
Vertauscht $a_1$ bis $a_k$ zyklisch, lässt den Rest.

$\tau  = (i j)$ ist 2-Zyklus bzw. \textit{Transposition}.

\subsubsection*{Signum $\sgn$}
$\sgn : (\S_n, \circ) \to (\{1,-1\},\cdot) \in \Hom$
$\sgn(\tau)=-1, \quad \sgn(\tau_1 \circ \dots \circ \tau_k)=(-1)^k$

\subsection*{Faktorgruppen}

\subsubsection*{Linksnebenklassen}
Gruppe $G$ und $U\le G$. $g,h\in G$.

$ g \sim h \iff g^{-1}h\in U,\quad \bar g = gU$\\
$ G/U = \{gU \mid g \in G \}, \quad |G:U| = |G/U| $

\subsubsection*{Satz von Lagrange}
Endl. Gruppe $G$ und $U \le G$.

$|G| = |U|\cdot|G:U|$

\subsubsection*{Faktorgruppe (G/U,\times)}
Abel. Gruppe $(G,\times)$ und $U \le G$.

$(G/U,\times)$ ist abel. Gruppe mit
Neutralem $\bar e$ und Inverse $\bar{g^{-1}}$.

	\section*{Ringe und Körper}

\subsection*{Ring mit Eins $(R,+,\cdot)$}
\begin{itemize}
	\item $(R,+)$ ist abel. Gruppe mit $0$
	\item $a\cdot(b\cdot c) = (a\cdot b)\cdot c$
	\item $\exists 1\in R: 1\cdot a = a \cdot 1 = a$
	\item $a \cdot (b+c)=a\cdot b + a\cdot c$ \\
		und $(b+c)\cdot a = b\cdot a + c \cdot a$
\end{itemize}
In kommutativen Ringen: $a\cdot b=b \cdot a$

\subsubsection*{Einheit in $R$}
$\exists a^{-1}\in R: a\cdot a^{-1}=a^{-1}\cdot a = 1$ \\
$\implies a$ ist Einheit \\
$R^* = \{a\in R \mid a \text{ ist Einheit}\}$ Gruppe

\subsubsection*{Unterring $S\le R$}
$1,~ a+b,~ -a,~ a\cdot b \in S$

\subsection*{Körper $(K,+,\cdot)$}
Komm. Ring mit Eins, wobei $K^* = K\setminus\{0\}$ \\
($(K\setminus\{0\},\cdot)$ ist abel. Gruppe)

\subsubsection*{Nullteilerfreiheit von Körpern}
$a,b\in K \setminus\{0\} \implies a\cdot b \neq 0$

\subsection*{$\Z_n=\Z / n\Z$}
$\Z_n$ ist ein komm. Ring mit Eins. \\
Wenn $n$ Primzahl, dann sogar Körper.

\subsection*{Der Polynomring $K[t]$}
$\displaystyle K[t] = \left\{\sum_{k=0}^n a_k\cdot t^k \mid a_k \in K, n\in\N \right\}$ \\
ist komm. Ring mit Eins $t^0$.

\subsubsection*{Koeffizientenvergleich}
$f=g \iff a_k = b_k \forall k$

\subsubsection*{Grad und Leitkoeffizient}
$\deg(f)=n$, $\deg(0)=-\infty$ \\
($f$ konstant, wenn $\deg(f)\le0$) \\
$\lc(f)=a_n$, $\lc(0)=0$

$\lc(f)=1 \lor f=0 \implies f$ normiert

\subsubsection*{Gradformeln}
\begin{itemize}
	\item $\deg(f+g) \le \max\{\deg(f),\deg(g)\}$
	\item $\deg(f\cdot g) = \deg(f)+\deg(g)$
\end{itemize}

\subsubsection*{Ideal $I$ von $K[t]$}
$I\subseteq K[t]$, $f,g\in I$ und $h\in K[t]$. \\
$f+g \in I$ und $h \cdot f \in I$

\subsubsection*{Division mit Rest}
$f,g \in K[t]\setminus\{0\}$
\begin{itemize}
	\item $\exists q,r \in K[t]: f=q\cdot g + r$ \\
		und $\deg(r) < \deg(g)$
	\item $\lambda\in K$ Nullstelle von $f$\\
		$\implies\exists q\in K[t]: f=q\cdot(t-\lambda)$
	\item Nullstellen $\le \deg(f)$
\end{itemize}

\subsubsection*{Fundamentalsatz der Algebra}
$\C$ ist algebraisch abgeschlossen, d.h.
jedes nicht-konstante Polynom in $\C[t]$ zerfällt über $\C$ in Linearfaktoren.

\subsubsection*{Irreduzible Polynome}
$f\in K[t]\setminus K$ ist irreduzibel, wenn \\
$f=g\cdot h \implies \deg(g)=0 \lor \deg(h)=0$

	\section*{Matrizen Basics}

$x,y\in K^n$, $A,B\in\Mat(m\times n,K)$, \\
$C\in\Mat(n\times p, K)$ und $\lambda\in K$.

\begin{itemize}
	\item $A(x+y)=Ax + Ay$\\ und $A(\lambda x) = \lambda Ax$
	\item $\lambda(A\circ C) = (\lambda A)\circ C = A\circ (\lambda C)$
	\item $f_{A+B}=f_A + f_B$
	\item $f_{\lambda A} = \lambda f_A$
	\item $f_{A \circ B} = f_A \circ f_B$
	\item Matrixmultiplikation ist assoziativ und distributiv
\end{itemize}

\subsection*{Allgemeine lineare Gruppe $\Gl_n(K)$}
$\Gl_n(K) = \{A \in \Mat_n(K) \mid A \text{ invertierbar}\}$ \\
ist eine Gruppe mit Neutralem $\mathbbm{1}_n$.

	\section*{Vektorräume}

Ein $K$-VR besteht aus der Menge $V$ und \\
$+: V\times V \to V : (x,y) \mapsto x + y$ und \\
$\cdot : K \times V \to V: (\lambda, x) \mapsto \lambda x$
\begin{itemize}
	\item $(V,+)$ ist abel. Gruppe
	\item $(\lambda + \mu)\cdot x = \lambda x + \mu x$ \\
		und $\lambda\cdot(x+y) = \lambda x + \lambda y$
	\item $(\lambda \cdot \mu)\cdot x = \lambda \cdot (\mu \cdot x)$
	\item $1\cdot x = x$
\end{itemize}

\subsection*{Unterraum $U \le V$}
$\forall \lambda \in K,~ x,y \in U: \lambda\cdot x\in U$ und $x+y\in U$

\subsubsection*{Durchschnitt von Unterräumen}
Der Durchschnitt von Unterräumen ist wieder ein Unterraum.

\subsection*{Lineare Hülle}
$M\subseteq V$. \\
$\Lin(M)=\langle M \rangle = $\\
$\{\lambda_1 x_1 + \dots + \lambda_r x_r \mid r\in\N, x_i \in M, \lambda_i \in K\}$ \\
$\Lin(M)\le V $

\subsection*{Summe zweier Unterräume}
VR $V$ und $U,U' \le V$.

$U + U' = \{u + u' \mid u\in U, u' \in U'\}$ \\
$= \Lin(U\cup U') \le V$

\subsubsection*{Direkte Summe}
$U = U_1 \oplus \dots \oplus U_n$, \\
wenn jedes $x$ eindeutig $x = u_1 + \dots + u_n$.

Insbesondere: \\
$W = U \oplus U' \Leftrightarrow W = U + U' \land U\cap U' = \{0\}$

	\section*{Lineare Abbildungen}
$\lambda,\mu \in K$ und $x,y \in V$.

$f(x+y)=f(x)+f(y)$ und $f(\lambda x) = \lambda f(x)$ \\
bzw. $f(\lambda x + \mu y) = \lambda f(x) + \mu f (y)$

\begin{itemize}
	\item $f(0_V)=0_W$ und $f(-x) = -f(x)$
	\item $f$ Isomorphismus $\implies f^{-1}$ linear
	\item $f \circ g$ linear
	\item $\Hom_K(V,W) \le W^V$
\end{itemize}

\subsection*{Matrizen und lineare Abbildungen}
$A\in\Mat(m\times n, K)$.\\
$\implies f_A: K^n\to K^m$ ist $K$-linear.

\subsection*{Kern und Bild als Unterräume}
$V,W$ VR und $f:V\to W$ linear.
\begin{itemize}
	\item $U\le V \implies f(U)\le W$
	\item $U\le W \implies f^{-1}(U) \le V$
	\item $\im(f)=f(V)\le W$
	\item $\Ker(f) = \{v \mid f(v)=0\} \le V$
	\item $f$ injektiv $\iff \Ker(f)=\{0_V\}$
\end{itemize}

\subsection*{Faktorraum}
Sei $V$ VR und $U \le V$.

$V/U$ ist mit verteterweisen $(+,\cdot)$ ein VR.

\subsubsection*{Komplemente}
$V$ ist VR, $U \le V$ und $U'$ Kompl. von $U$.

$\pi_{\mid}: U' \to V/u: x \mapsto \bar x$ \\
ist Isomorphismus.

\subsection*{Homomorphiesatz}
$f: V \to W$ linear.

$\implies \bar f: V/\Ker(f) \to \Im(f): \bar x \mapsto f(x)$ \\
ist Isomorphismus.

	\section*{Basen von Vektorräumen}

\subsection*{Lineare Unabhängigkeit}
$(x_1, \dots, x_n)$ ist linear unabhängig, wenn \\
$\lambda_1\cdot x_1 + \dots + \lambda_n \cdot x_n \implies \lambda_1 = \dots = \lambda_n = 0$

\subsubsection*{Kriterien für lineare Abhängigkeit}
Wenn $0\in F$ oder $\exists x\in F$, das Linearkombination anderer Vektoren in $F$ ist, dann ist $F$ linear abhängig.

\subsection*{EZS und Basis}
$V$ ist VR, $F$ Familie aus $V$.

$F$ ist EZS von $V$ $\iff V=\Lin(F)$.

$F$ ist Basis $\iff F$ EZS und lin. unabh. 

\subsubsection*{Charakterisierung von Basen}
$B$ ist Basis bedeutet:
\begin{itemize}
	\item $B$ ist minimales EZS
	\item $B$ ist maximal lin. unabhängig
\end{itemize}

\subsection*{Austauschen in endl. VR}
\subsubsection*{Austauschlemma}
$B=(x_1,\dots,x_n)$ Basis, \\
$\displaystyle y=\sum_{i=1}^n \lambda_i x_i$ und $\lambda_j \neq 0$.

$\implies (x_1,\dots,x_{j-1},y,x_{j+1},\dots,x_n)$ Basis.

\subsubsection*{Austauschsatz von Steinitz}
$(x_1,\dots,x_n)$ Basis, $(y_1,\dots,y_r)$ lin. unabh.

$\implies x_1,\dots,x_n$ lassen sich so umnummerieren,
dass $(y_1,\dots,y_r,x_{r+1},\dots,x_n)$ Basis.

\subsubsection*{Basisergänzungssatz}
Wenn $F=(y_1,\dots,y_r)$ lin. unabhängig in $V$ (endlich),
dann kann $F$ zu einer Basis von $V$ ergänzt werden.

(Steinitz mit einer Basis und $F$)

\subsection*{Gleichmächtige Basen}
Alle Basen von $V$ sind gleichmächtig.

	\section*{Dimension von VR}

$\dim_K(V)=\begin{cases}
	n, & \text{Basis hat $n$ Elemente}\\
	\infty & \text{nicht endlich erzeugt}
\end{cases}$

\subsection*{Dimension als Invariante}
$V,W$ sind $K$-Vektorräume.

$V \cong W \iff \dim_K(V) = \dim_K(W)$

\subsubsection*{Lin. Abb. zwischen gleichdim. VR}
$V,W$ VR mit $\dim_K(V)=\dim_K(W)$, \\
$f:V\to W$ linear, dann für $f$: \\
inj. $\iff$ surj. $\iff$ bij.

\subsection*{Dimensionsformeln}
Ist $\dim_K(V)<\infty$ und $U \le V$:

$\dim_K(U) \le \dim_K(V)$ und \\
$U = V \iff \dim_K(U) = \dim_K(V)$

\subsubsection*{Für Unterräume}
$U,U' \le V$, dann:
$\dim_K(U+U')$ \\ $=\dim_K(U)+\dim_K(U')-\dim_K(U\cap U')$

\subsubsection*{Für Komplemente}
Für $U,U' \le V$ sind äquivalent:
\begin{itemize}
	\item $V=U \oplus U'$
	\item $V=U+U'$ und \\ $\dim_K V = \dim_K U + \dim_K U'$
	\item $U \cap U'$ und \\ $\dim_K V = \dim_K U + \dim_K U'$
\end{itemize}

\subsubsection*{Für Faktorräume}
$U \le V$, dann:

$\dim_K(V/U) = \dim_K(V)-\dim_K(U)$

\subsubsection*{Für lineare Abbildungen}
$f: V \to W$ linear, dann:

$\dim_K(V) = \dim_K(\Ker(f)) + \dim_K(\im(f))$

	\section*{Matrixdarstellung lin. Abb.}

\subsection*{Koordinatenvektor $M_B(x)$}
Basis $B=(b_1,\dots,b_n)$.\\
Vektor $x=(\lambda_1 b_1,\dots,\lambda_n b_n)$. \\
$\implies M_B(x) = (\lambda_1,\dots,\lambda_n)^t$

\subsection*{Matrixdarstellung $M_D^B(f)$}
Basen $B,D$ von $V$ bzw. $W$. \\
$f:V\to W$ linear.

$M_D^B(f)=(M_D(f(b_1)),\dots,M_D(f(b_n)))$

Insb.: $M_D^B: \Hom_K(V,W) \to \Mat(m \times n, K)$
ist Isomorphismus.

\subsection*{Rechnen mit Koordinaten}
\subsubsection*{Bild eines Vektors}
$f: V\to W, $ linear. $x\in V$.

$M_D(f(x))=M_D^B \circ M_B(x)$

\subsubsection*{Komposition}
$f: U\to V, g: V\to W$ linear. \\
$B,C,D$ Basen von $U,V,W$.

$M_D^B(g \circ f) = M_D^C(g) \circ M^B_C(f)$

	\section*{Basiswechsel}

\subsection*{Koordinatentransformationsmatrix $T_{B'}^B$}
$B,B'$ Basen von $V$.

$T_{B'}^B=M_{B'}^B(\id_V)=\big(M_{B'}(b_1),\dots,M_{B'}(b_n)\big)$

\subsubsection*{Invertierbarkeit}
$\left(T_{B'}^B\right)^{-1} = T_B^{B'}$

\subsection*{Anwendung}

\subsubsection*{Matrixdarstellung}
$M_{D'}^{B'}(f) = T_{D'}^D \circ M_D^B(f) \circ T_B^{B'}$

\subsubsection*{Endomorphismen}
$T = T_B^{B'}$ und $f \in\End_K(V)$.

$M_{B'}^{B'}(f) = T^{-1} \circ M_B^B(f) \circ T$

	\section*{Rang von Matrizen}

$f \in \Hom_K(V,W), A\in\Mat(m \times n, K)$.

$\rang(f) := \dim_K(\im(f))$ \\
$\rang(A) := \rang(f_A)$

\subsection*{Invertierbare Matrizen}
$A \in\Mat_n(K)$ inv.bar $\iff \rang(A) = n$

\subsection*{Rang von Matrixdarstellung}
$f\in\Hom_K(V,W)$. \\
$\rang(f) = \rang(M_D^B(f))$

\subsection*{Normalform bzgl. Äquivalenz}

\subsubsection*{Matrixdarstellung}
$f \in\Hom_K(V,W)$. \\
$\exists B,D$, so dass $M_D^B(f)$ in Normalform.

\subsubsection*{Matrix}
$A\in\Mat(m \times n, K)$. \\
$\exists S \in \Gl_m(K), T\in\Gl_n(K)$, so dass
$S \circ A \circ T$ in Normalform.

\subsection*{Zeilen- und Spaltenrang}
\subsubsection*{Transponierte und Inverse}
$A\in\Mat_n(K) \text{ inv.bar} \iff A^t \text{ inv.bar}$ \\
$\implies (A^t)^{-1} = (A^{-1})^t$

\subsubsection*{Rang der Transponierten}
$A\in\Mat(m \times n, K)$ \\
$\implies \rang(A) = \rang(A^t)$

	\section*{Gauß-Algorithmus}

\subsection*{Elementare Operationen}
Elementare Operationen für jeweils Zeilen oder Spalten:
\begin{itemize}
	\item Zeile/Spalte tauschen
	\item Zeile/Sp. mit $\lambda\neq0$ multiplizieren
	\item Zeile/Spalte mit Vielfachem einer anderen addieren
\end{itemize}

\subsection*{Zeilen-Stufen-Form}
Spaltenweise von links nach rechts unter Pivots Nullen erzeugen.

Für reduzierte ZSF: Pivots auf 1 skalieren
und dann von rechts nach links Nullen über Pivots erzeugen.

\subsection*{Bestimmung des Rangs}
Anzahl der Nicht-Nullzeilen einer ZSF.

\subsection*{Bestimmung der Inversen}
\begin{enumerate}
	\item $C = (A, \mathbbm{1}_n)$
	\item $C'=(A',B)=\rZSF(C)$
	\item $\rang(A')=n \implies $ Inverse ist $B$
\end{enumerate}

\subsection*{Bestimmung der Normalform}
\begin{enumerate}
	\item $A$ mit Zeilenoperationen in $\rZSF$,
		analog $\mathbbm{1}_m$ in $S$.
	\item $\rZSF(A)$ mit Spaltenoperationen in Normalform,
		analog $\mathbbm{1}_n$ in $T$.
	\item $S \circ A \circ T = $ Normalform
\end{enumerate}

\subsection*{Basisberechnung}
$F$ ist EZS von $U \subseteq K^n$.
\begin{enumerate}
	\item Vektoren aus $F$ als Zeilen in $A$
	\item $A$ in ZSF
	\item Erste $\rang(A)$ Zeilen sind Basis
\end{enumerate}

\subsection*{Injektiv/Surjektiv/Bijektiv}
\begin{itemize}
	\item $\rang(A)=m=n$: $f_A$ bijektiv
	\item $\rang(A)=m<n$: $f_A$ surjektiv
	\item $\rang(A)=n<m$: $f_A$ injektiv
\end{itemize}

\subsection*{Summe zweier Unterräume}
$F,G$ EZS von $U,U' \le K^n$.

Basis von $U+U'=\langle F \cup G \rangle$ wie vorher.

\subsection*{Test auf lin. Unabhängigkeit}
$F$ Familie von $m$ Vektoren aus $K^n$.

\begin{enumerate}
	\item $F$ als Spalten in $A$
	\item $\rang(A)=m \iff $ lin. unabh.
\end{enumerate}

\subsection*{Bild von $f_A$ bestimmen}
\begin{enumerate}
	\item ZSF von $A^t$ berechnen
	\item Erste $\rang(A)$ Zeilen sind Basis von $\im(f)$
\end{enumerate}

	\section*{Lineare Gleichungssysteme}

\subsection*{Struktur von Lös$(A,b)$}
$c$ eine Lösung von $Ax=b$.

Lös$(A,b)=c+$ Lös$(A,0)$.

\subsection*{Lösbarkeit}
$Ax=b$ lösbar $\Leftrightarrow \rang(A) = \rang(A \mid b)$

\subsection*{Gauß-Alg. zur Lösung eines LGS}
\begin{enumerate}
	\item $(A'|b')=\rZSF(A|b)$, $r=\rang(A)$
	\item $b'_{r+1}\neq 0 \implies$ nicht lösbar
	\item Pivots auf Diagonale bringen (Nullzeilen/-spalten hinzufügen oder streichen) $\rightarrow (A''|b'')$
	\item Ersetze 0 auf Diagonale mit $-1$.
	\item $c=b''$, Spalten von $A''$ mit $-1$ sind Basis von Lös$(A,0)$.
\end{enumerate}

\subsection*{Kern von $f_A$ bestimmen}
Löse LGS $Ax=0$.

\subsection*{Transformationsmatrix $T_{B'}^B$ bestimmen}
$B=(b_1,\dots,b_n)$ und $B'=(b'_1,\dots,b'_n)$.

\begin{enumerate}
	\item $b'_1,\dots,b'_n,b_1,\dots,b_n$ Spalten in $A$.
	\item Letzte $n$ Spalten von $\rZSF(A)$: $T_{B'}^B$.
\end{enumerate}

\subsection*{Matrixdarstellung $M_D^B(f)$ bestimmen}
$f:K^n\to K^m$, $B,D$ Basen von $K^n,K^m$.
\begin{enumerate}
	\item $d_1,\dots,d_m,f(b_1),\dots,f(b_n)$ Sp. in $A$.
	\item Letzte $n$ Sp. von $\rZSF(A)$: $M_D^B(f)$
\end{enumerate}

\subsection*{Austauschverfahren von Steinitz}
$B$ Basis, $F=(y_1,\dots,y_r)$ lin. unabh. in $V$.

Konstruiere Basis $B'$, die $F$ enthält: \\
Schreibe $B$ in $B'$, dann wiederhole für $i=1,\dots,r$:
\begin{enumerate}
	\item $B'$ als Spalten in $A$
	\item $(A,y_i)$ in $\rZSF$, suche in der letzten Spalte den ersten Eintrag $\neq 0$.
	\item Streiche entsprechenden Vektor aus $B'$ und füge $y_i$ als letzten hinzu.
\end{enumerate}

\subsection*{Gleichungen eines Unterraums}
$F$ Familie im $K^n$.

\begin{enumerate}
	\item $F$ als Zeilen in $B\in\Mat(m \times n,K)$.
	\item Bestimme Basis $(y_1,\dots,y_k)$ von $\Ker(f_B)=$ Lös$(B,0)$.
	\item $y_1,\dots,y_k$ als Zeilen in $A$, \\
		$\implies$ Lös$(A,0)=\Lin(F)$ 
\end{enumerate}

\subsection*{Durchschnitt zweier Unterräume}
$F,G$ Familien im $K^n$.
\begin{enumerate}
	\item Bestimme jeweils Gleichungen für $F$ und $G$ (siehe oben).
	\item Schreibe alle Zeilen in $A$.
	\item Basis von $\Ker(f_A)=$ Lös$(A,0)$ ist Basis des Durchschnitts.
\end{enumerate}

	\section*{Determinante}
$A\in\Mat_n(K)$

\subsection*{Leibniz-Formel}
$\displaystyle \det(A) := \sum_{\sigma\in\S_n} \sgn(\sigma)\cdot a_{1\sigma(1)}\cdot\dots\cdots a_{n\sigma(n)}$

\subsection*{Kleine Matrizen}
$2\times2$: $a\cdot d - b\cdot c$ \\
$3\times3$: Sarrus

\subsection*{Dreiecksmatrizen}
Für obere/untere Dreiecksmatrizen ist die Determinante
das Produkt der Diagonaleinträge.

\subsection*{Transponierte}
$\det(A) = \det(A^t)$

\subsection*{Determinante als Volumenform}

\subsubsection*{Multilineare Abbildungen}
$f:V^n\to W$ multilinear, wenn
$f$ in jedem Argument linear ist.

\subsubsection*{Alternierende multi. Abbildungen}
Wenn $f(x_1,\dots,x_n)=0$, sobald zwei Argumente gleich sind,
ist $f$ alternierend.

$f(x_{\sigma(1)},\dots,x_{\sigma(n)}) = \sgn(\sigma)\cdot f(x_1,\dots,x_n)$

\subsubsection*{Determinante als Volumenform}
$\det : \Mat_n(K)\to K$ ist alternierend multilinear
mit $\det(\mathbbm{1}_n)=1$.

Ist $f: \Mat_n(K)\to K$ alt. mult.,\\
dann $f(A)=f(\mathbbm{1}_n) \cdot \det(A)$

\subsection*{Determinante und Gauß}
Es gilt bei Änderung von $A$ auf $A'$:
\begin{itemize}
	\item Zeilen/Spalten tauschen:\\ $\det(A')=-\det(A)$
	\item Zeile/Spalte mit $\lambda$ skalieren:\\ $\det(A')=\lambda\det(A)$
	\item Zeilen/Sp. aufeinander addieren:\\ $\det(A')=\det(A)$
	\item Nullzeile/-spalte: $\det(A)=0$
	\item Gleiche Zeilen/Sp.: $\det(A)=0$
\end{itemize}

\subsection*{Determinantenmultiplikationssatz}
$\det(A \circ B)=\det(A) \cdot \det(B)$

\subsection*{Invertierbarkeit}
$A$ inv.bar $\iff \det(A) \neq 0$. \\
Dann $\det(A^{-1}) = (\det(A))^{-1}$

\subsection*{Kästchensatz}
Wenn $A=\mat{B & C \\ 0 & D}$
mit $B,D$ quadratisch.

$\det(A)=\det(B)\cdot\det(D)$

\subsection*{Adjunkte}
$A \in \Mat_n(K), n \ge 2$

\subsubsection*{Ersetzungs-/Streichungsmatrix}
$A_i(b)$ ersetzt die $i$-te Spalte von $A$ mit $b$.

$A_{ji}$ ist $A$ ohne $j$-te Zeile und $i$-te Spalte.

\subsubsection*{Kofaktor und Adjunkte}
$a^\#_{ij} := (-1)^{i+j}\cdot\det(A_{ji})$ \\
$A^\# := (a^\#_{ij})\in\Mat_n(K)$

\subsubsection*{Satz über die Adjunkte}
$A^\# \circ A = A \circ A^\# = \det(A) \cdot \mathbbm{1}_n$

\subsubsection*{Inverse über Adjunkte}
$A$ inv.bar $\displaystyle \implies A^{-1}=\frac{1}{\det(A)}\cdot A^\#$

\subsection*{Laplacescher Entwicklungssatz}
Entwickle $\det$ nach $i$-ter Zeile: \\
$\displaystyle \det(A)=\sum_{j=1}^n(-1)^{i+j}\cdot a_{ij} \cdot \det(A_{ij})$ \\
(Ersetze Laufindex durch $i$ für $j$-te Spalte)

Vorzeichen: Beachte Schachbrettmuster.

\subsection*{Cramersche Regel}
$A\in\Mat_n(K)$ inv.bar und $b\in K^n$.

$Ax=b$ hat eindtg. Lös. $x=(x_1,\dots,x_n)^t$.

$x_i = \frac{1}{\det(A)}\cdot\det(A_i(b))$

	\section*{Eigenwerte}

\subsection*{Charakteristisches Polynom $\chi_A$}
$\chi_A := \det(t\cdot\mathbbm{1}_n - A)\in K[t]$

\subsubsection*{Spur einer Matrix}
$\Spur(A)=$ Summe der Diagonaleinträge.

\subsubsection*{Form von $\chi_A$}
$\chi_A=t^n + \alpha_{n-1}t^{n-1}+\dots+\alpha_1 t + \alpha_0 \in K[t]$
mit $\alpha_{n-1}=-\Spur(A)$ und $\alpha_0=(-1)^n\cdot\det(A)$.

\subsection*{Konjugation quadratischer Matrizen}
$A,B\in\Mat_n(K)$ konjugiert bzw. ähnlich, wenn
$\exists T\in\Gl_n(K): B=T^{-1}\circ A \circ T$

\subsubsection*{Eigenschaften}
$A,B\in\Mat_n(K)$ konjugiert: \\
$\chi_A=\chi_B,\quad \det(A)=\det(B)$ \\ und $\Spur(A)=\Spur(B)$

\subsection*{Endomorphismen}
$\dim_K(V) \ge 1$, Basis $B$ und $f\in\End_K(V)$.

$\chi_f = \chi_{M_B^B(f)},\quad \det(f)=\det(M_B^B(f))$\\
und $\Spur(f)=\Spur(M_B^B(f))$

\subsection*{Eigenwerte und -vektoren}
$\lambda \in K$, $f\in\End_K(V)$ bzw. $A\in\Mat_n(K)$.

$\exists 0\neq x\in V: f(x)=\lambda x$ \\
bzw. $\exists 0\neq x\in K^n: Ax = \lambda x$

$\implies \lambda$ ist Eigenwert, $x$ ist Eigenvektor.

\subsubsection*{Eigenräume}
$\Eig(f,\lambda) := \{y\in V \mid f(y)=\lambda y\}$ \\
$\Eig(A,\lambda) := \{y\in K^n \mid Ay=\lambda y\}$

\subsection*{Eigenwerte und charakt. Polynom}
Die Eigenwerte von $f$ bzw. $A$ sind genau die Nullstellen von $\chi_f$ bzw. $\chi_A$ in $K$.

\subsection*{Dreiecksmatrizen}
Für obere/untere Dreiecksmatr. sind die Diagonaleinträge genau die Eigenwerte.
$\chi_A=(t-a_{11})\cdot\dots\cdot(t-a_{nn})$

\subsection*{Linear unabhängige Eigenvektoren}
$x_1,\dots,x_n$ Eigenvektoren zu paarweise verschiedenen Eigenwerten.

Dann ist $(x_1,\dots,x_n)$ linear unabhängig.

	\section*{Diagonalisierbarkeit}

$f\in\End_K(V)$ und $A\in\Mat_n(K)$.

$\exists B$ Basis: $M_B^B(f)$ Diagonalmatrix bzw. \\
$\exists T\in\Gl_n(K)$: $T^{-1}\circ A \circ T$ Diagonalmatrix. \\
$\implies$ $f$ bzw. $A$ ist diagonalisierbar.

\subsection*{Endomorphismen}
Gleichbedeutend:
\begin{itemize}
	\item $f$ diagonalisierbar
	\item $V$ hat Basis aus Eigenvekt. von $f$
	\item $\lambda_1,\dots,\lambda_r$ pw. versch. EW von $f$ \\
		$\implies V = \bigoplus \Eig(f,\lambda_i)$ \\
		bzw. $\dim_K(V)=\sum \dim_K\Eig(f, \lambda_i)$
\end{itemize}

\subsection*{Quadratische Matrizen}
Gleichbedeutend:
\begin{itemize}
	\item $A$ diagonalisierbar
	\item $K^n$ hat Basis aus Eigenvekt. von $A$
	\item $\lambda_1,\dots,\lambda_r$ pw. versch. EW von $A$ \\
		$\implies K^n = \bigoplus \Eig(A,\lambda_i)$ \\
		bzw. $n=\sum \dim_K\Eig(A, \lambda_i)$
\end{itemize}

\subsection*{Einfaches Kriterium}
Hat $A\in\Mat_n(K)$ genau $n$ pw. versch. EW, ist $A$ diagonalisierbar.

\subsection*{Diagonalisierung}
$A\in\Mat_n(K)$.
\begin{enumerate}
	\item Berechne und faktorisiere $\chi_A$.
	\item Ein Faktor nicht linear \\
		$\rightarrow$ $A$ nicht diagonalisierbar.
	\item Bestimme EW $\lambda_1,\dots,\lambda_r$ mit ihren algeb. Vielfachheiten $n_i$.
	\item Bestimme Eigenraumbasen mit Lös$(\lambda_i\mathbbm{1}_n-A,0)$.
	\item $n_i$ \& Dim. stimmen nicht überein \\ $\rightarrow$ $A$ nicht diagonalisierbar. 
	\item Basen als Spalten in $T$ und $\lambda_i$ entsprechend auf Diagonale von $D$ \\
		$\implies T^{-1}\circ A \circ T = D$
\end{enumerate}

	\section*{Euklidische \& unitäre Räume}
$\K\in\{\R,\C\}$

\subsection*{Komplexe Zahlen $\C$}
$z=x+iy$
\begin{itemize}
	\item $|z| = \sqrt{x^2 + y^2}$
	\item $\overline{z} = x - iy$
\end{itemize}

\subsection*{Skalarprodukt $\langle\cdot,\cdot\rangle$}
Für $\langle\cdot,\cdot\rangle: V \times V \to \K$ muss gelten:
\begin{itemize}
	\item $\langle x,\lambda y + \mu z \rangle = \lambda\cdot\langle x,y\rangle + \mu\cdot\langle x,z\rangle$
	\item $\langle x,y \rangle = \overline{\langle y,x \rangle}$
	\item $\forall x\neq 0: 0 < \langle x,x \rangle \in \R$
\end{itemize}

\subsubsection*{Standardskalarprodukt}
$\displaystyle \langle\cdot,\cdot\rangle: \K^n \times \K^n \to \K: (x,y)\mapsto \sum_{i=1}^n \overline{x_i}\cdot y_i$

\subsubsection*{Euklidischer/Unitärer Raum}
Euklidischer Raum: $\R$-VR mit Skalarpr. \\
Unitärer Raum: $\C$-VR mit Skalarprodukt.

\subsubsection*{Cauchy-Schwarzsche Ungleichung}
Im eukl./unit. Raum $V$ gilt: \\
$|\langle x,y\rangle| \le \sqrt{\langle x,x \rangle}\cdot\sqrt{\langle y,y \rangle}$

\subsection*{Norm $\|\cdot\|$}
Für $\|\cdot\|:V\to\R_{\ge 0}$ muss gelten:
\begin{itemize}
	\item $\|x\| = 0 \iff x = 0$
	\item $\|\lambda x\|=|\lambda|\cdot\|x\|$
	\item $\|x+y\| \le \|x\| + \|y\|$
\end{itemize}

\subsubsection*{Euklidische Norm}
In einem eukl./unit. Raum $V$, definiere:
$\|\cdot\|: V\to\R_{\ge0}:x\mapsto \sqrt{\langle x,x \rangle}$

\end{multicols*}
\end{document}
