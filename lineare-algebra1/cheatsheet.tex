\documentclass[10pt,landscape,a4paper]{article}
\usepackage[ngerman]{babel}
\usepackage[T1]{fontenc}
\usepackage[dvipsnames]{xcolor}
%\usepackage[LY1,T1]{fontenc}
\usepackage{tikz}
\usetikzlibrary{shapes,positioning,arrows,fit,calc,graphs,graphs.standard}
\usepackage[nosf]{kpfonts}
\usepackage{multicol}
\usepackage{wrapfig}
\usepackage[top=2mm,bottom=3mm,left=2mm,right=2mm,headsep=1mm,includehead]{geometry}
\usepackage[framemethod=tikz]{mdframed}
\usepackage{microtype}
\usepackage{pdfpages}
\usepackage{csquotes}
\usepackage{booktabs}

\usepackage{amsmath}
\usepackage{bbm}
\newcommand{\N}{\mathbb{N}}
\newcommand{\Z}{\mathbb{Z}}
\newcommand{\R}{\mathbb{R}}
\newcommand{\C}{\mathbb{C}}
\newcommand{\F}{\mathbb{F}}
\renewcommand{\S}{\mathbb{S}}
\DeclareMathOperator{\im}{Im}
\DeclareMathOperator{\Ker}{Ker}
\DeclareMathOperator{\Sym}{Sym}
\DeclareMathOperator{\sgn}{sgn}
\DeclareMathOperator{\Hom}{Hom}
\DeclareMathOperator{\End}{End}
\DeclareMathOperator{\lc}{lc}
\DeclareMathOperator{\Mat}{Mat}
\DeclareMathOperator{\Gl}{Gl}
\DeclareMathOperator{\Lin}{Lin}
\DeclareMathOperator{\id}{id}
\DeclareMathOperator{\rang}{rang}

\usepackage{blindtext}

\usepackage{titlesec}
\newcommand{\sectionrule}{ \par \hrule \nopagebreak \bigskip }

\titleformat{\section}{\sectionrule\large\color{Mahogany}}{\thesection}{1em}{\scshape}
\titleformat{\subsection}{\color{RoyalPurple}}{\thesubsection}{1em}{}
\titleformat{\subsubsection}{}{\thesubsubsection}{1em}{\itshape}

\titlespacing*{\section}{0cm}{0.4\baselineskip}{0.4\baselineskip}
\titlespacing*{\subsection}{0cm}{0.3\baselineskip}{0.3\baselineskip}
\titlespacing*{\subsubsection}{0cm}{0.2\baselineskip}{0.2\baselineskip}


\usepackage{hyperref}
\newcommand\nohyper[1]{\begin{NoHyper}#1\end{NoHyper}}

\usepackage{lastpage,fancyhdr}
\pagestyle{fancy}
\fancyhf{}
\fancyhead[L]{Rasmus Buurman}
\fancyhead[C]{\textsc{Lineare Algebra I}}
\fancyhead[R]{Seite \thepage{}/\nohyper{\pageref{LastPage}}}
\renewcommand\headrulewidth{0.5pt}

\let\bar\overline

\begin{document}
\small
\begin{multicols*}{5}
	\section*{Gruppen \& Homomorphismen}

\subsection*{Gruppe $(G,\times)$}
\begin{itemize}
	\item $(g \times h) \times k = g \times (h \times k)$
	\item $\exists e \in G: \forall g\in G: e\times g = g$
	\item $\forall g\in G: \exists g^{-1}\in G: g^{-1} \times g = e$
\end{itemize}
In abelschen Gruppen: $g\times h = h \times g$

\subsubsection*{Untergruppe $U\le G$}
$\forall u,v\in U: u\times v \in U$ und $u^{-1}\in U$

\subsection*{Erzeugnis}
Gruppe $G$ und $M \subseteq G$.

$\displaystyle \langle M \rangle := \bigcap_{M\subseteq U \le G}U$ \\
$= \{g_1^{\alpha_1}\times\cdots\times g_n^{\alpha_n} \mid g_i\in M,\alpha_i\in\Z\} $

\subsection*{Gruppenhomomorphismus $\varphi:G \to H$}
$(G, \times)$ und $(H, *)$ Gruppen. \\
$\forall g,g'\in G: \varphi(g\times g')=\varphi(g) * \varphi(g')$

\subsubsection*{Bild und Kern}
\begin{itemize}
	\item $\im(\varphi):=\varphi(G) \le H$
	\item $\Ker(\varphi) := \varphi^{-1}(e_H) \le G$
	\item $\varphi$ injektiv $\iff \Ker(\varphi)=\{e_G\}$
\end{itemize}

\subsection*{Die symmetrische Gruppe $\S_n$}
$\S_n := \Sym(\{1,\dots,n\})$\\
$=\{\sigma \mid \sigma \text{ ist Permutation}\}$\\
($\sigma : \{1,\dots,n\}\to\{1,\dots,n\}$ bijektiv)

\subsubsection*{Zyklen und Transpositionen}
$\sigma = (a_1 \dots a_k)$ ist \textit{k-Zyklus}:
Vertauscht $a_1$ bis $a_k$ zyklisch, lässt den Rest.

$\tau  = (i j)$ ist 2-Zyklus bzw. \textit{Transposition}.

\subsubsection*{Signum $\sgn$}
$\sgn : (\S_n, \circ) \to (\{1,-1\},\cdot) \in \Hom$
$\sgn(\tau)=-1, \quad \sgn(\tau_1 \circ \dots \circ \tau_k)=(-1)^k$

\subsection*{Faktorgruppen}

\subsubsection*{Linksnebenklassen}
Gruppe $G$ und $U\le G$. $g,h\in G$.

$ g \sim h \iff g^{-1}h\in U,\quad \bar g = gU$\\
$ G/U = \{gU \mid g \in G \}, \quad |G:U| = |G/U| $

\subsubsection*{Satz von Lagrange}
Endl. Gruppe $G$ und $U \le G$.

$|G| = |U|\cdot|G:U|$

\subsubsection*{Faktorgruppe (G/U,\times)}
Abel. Gruppe $(G,\times)$ und $U \le G$.

$(G/U,\times)$ ist abel. Gruppe mit
Neutralem $\bar e$ und Inverse $\bar{g^{-1}}$.

	\section*{Ringe und Körper}

\subsection*{Ring mit Eins $(R,+,\cdot)$}
\begin{itemize}
	\item $(R,+)$ ist abel. Gruppe mit $0$
	\item $a\cdot(b\cdot c) = (a\cdot b)\cdot c$
	\item $\exists 1\in R: 1\cdot a = a \cdot 1 = a$
	\item $a \cdot (b+c)=a\cdot b + a\cdot c$ \\
		und $(b+c)\cdot a = b\cdot a + c \cdot a$
\end{itemize}
In kommutativen Ringen: $a\cdot b=b \cdot a$

\subsubsection*{Einheit in $R$}
$\exists a^{-1}\in R: a\cdot a^{-1}=a^{-1}\cdot a = 1$ \\
$\implies a$ ist Einheit \\
$R^* = \{a\in R \mid a \text{ ist Einheit}\}$ Gruppe

\subsubsection*{Unterring $S\le R$}
$1,~ a+b,~ -a,~ a\cdot b \in S$

\subsection*{Körper $(K,+,\cdot)$}
Komm. Ring mit Eins, wobei $K^* = K\setminus\{0\}$ \\
($(K\setminus\{0\},\cdot)$ ist abel. Gruppe)

\subsubsection*{Nullteilerfreiheit von Körpern}
$a,b\in K \setminus\{0\} \implies a\cdot b \neq 0$

\subsection*{$\Z_n=\Z / n\Z$}
$\Z_n$ ist ein komm. Ring mit Eins. \\
Wenn $n$ Primzahl, dann sogar Körper.

\subsection*{Der Polynomring $K[t]$}
$\displaystyle K[t] = \left\{\sum_{k=0}^n a_k\cdot t^k \mid a_k \in K, n\in\N \right\}$ \\
ist komm. Ring mit Eins $t^0$.

\subsubsection*{Koeffizientenvergleich}
$f=g \iff a_k = b_k \forall k$

\subsubsection*{Grad und Leitkoeffizient}
$\deg(f)=n$, $\deg(0)=-\infty$ \\
($f$ konstant, wenn $\deg(f)\le0$) \\
$\lc(f)=a_n$, $\lc(0)=0$

$\lc(f)=1 \lor f=0 \implies f$ normiert

\subsubsection*{Gradformeln}
\begin{itemize}
	\item $\deg(f+g) \le \max\{\deg(f),\deg(g)\}$
	\item $\deg(f\cdot g) = \deg(f)+\deg(g)$
\end{itemize}

\subsubsection*{Ideal $I$ von $K[t]$}
$I\subseteq K[t]$, $f,g\in I$ und $h\in K[t]$. \\
$f+g \in I$ und $h \cdot f \in I$

\subsubsection*{Division mit Rest}
$f,g \in K[t]\setminus\{0\}$
\begin{itemize}
	\item $\exists q,r \in K[t]: f=q\cdot g + r$ \\
		und $\deg(r) < \deg(g)$
	\item $\lambda\in K$ Nullstelle von $f$\\
		$\implies\exists q\in K[t]: f=q\cdot(t-\lambda)$
	\item Nullstellen $\le \deg(f)$
\end{itemize}

\subsubsection*{Fundamentalsatz der Algebra}
$\C$ ist algebraisch abgeschlossen, d.h.
jedes nicht-konstante Polynom in $\C[t]$ zerfällt über $\C$ in Linearfaktoren.

\subsubsection*{Irreduzible Polynome}
$f\in K[t]\setminus K$ ist irreduzibel, wenn \\
$f=g\cdot h \implies \deg(g)=0 \lor \deg(h)=0$

	\section*{Matrizen Basics}

$x,y\in K^n$, $A,B\in\Mat(m\times n,K)$, \\
$C\in\Mat(n\times p, K)$ und $\lambda\in K$.

\begin{itemize}
	\item $A(x+y)=Ax + Ay$\\ und $A(\lambda x) = \lambda Ax$
	\item $\lambda(A\circ C) = (\lambda A)\circ C = A\circ (\lambda C)$
	\item $f_{A+B}=f_A + f_B$
	\item $f_{\lambda A} = \lambda f_A$
	\item $f_{A \circ B} = f_A \circ f_B$
	\item Matrixmultiplikation ist assoziativ und distributiv
\end{itemize}

\subsection*{Allgemeine lineare Gruppe $\Gl_n(K)$}
$\Gl_n(K) = \{A \in \Mat_n(K) \mid A \text{ invertierbar}\}$ \\
ist eine Gruppe mit Neutralem $\mathbbm{1}_n$.

	\section*{Vektorräume}

Ein $K$-VR besteht aus der Menge $V$ und \\
$+: V\times V \to V : (x,y) \mapsto x + y$ und \\
$\cdot : K \times V \to V: (\lambda, x) \mapsto \lambda x$
\begin{itemize}
	\item $(V,+)$ ist abel. Gruppe
	\item $(\lambda + \mu)\cdot x = \lambda x + \mu x$ \\
		und $\lambda\cdot(x+y) = \lambda x + \lambda y$
	\item $(\lambda \cdot \mu)\cdot x = \lambda \cdot (\mu \cdot x)$
	\item $1\cdot x = x$
\end{itemize}

\subsection*{Unterraum $U \le V$}
$\forall \lambda \in K,~ x,y \in U: \lambda\cdot x\in U$ und $x+y\in U$

\subsubsection*{Durchschnitt von Unterräumen}
Der Durchschnitt von Unterräumen ist wieder ein Unterraum.

\subsection*{Lineare Hülle}
$M\subseteq V$. \\
$\Lin(M)=\langle M \rangle = $\\
$\{\lambda_1 x_1 + \dots + \lambda_r x_r \mid r\in\N, x_i \in M, \lambda_i \in K\}$ \\
$\Lin(M)\le V $

\subsection*{Summe zweier Unterräume}
VR $V$ und $U,U' \le V$.

$U + U' = \{u + u' \mid u\in U, u' \in U'\}$ \\
$= \Lin(U\cup U') \le V$

\subsubsection*{Direkte Summe}
$U = U_1 \oplus \dots \oplus U_n$, \\
wenn jedes $x$ eindeutig $x = u_1 + \dots + u_n$.

Insbesondere: \\
$W = U \oplus U' \Leftrightarrow W = U + U' \land U\cap U' = \{0\}$

	\section*{Lineare Abbildungen}
$\lambda,\mu \in K$ und $x,y \in V$.

$f(x+y)=f(x)+f(y)$ und $f(\lambda x) = \lambda f(x)$ \\
bzw. $f(\lambda x + \mu y) = \lambda f(x) + \mu f (y)$

\begin{itemize}
	\item $f(0_V)=0_W$ und $f(-x) = -f(x)$
	\item $f$ Isomorphismus $\implies f^{-1}$ linear
	\item $f \circ g$ linear
	\item $\Hom_K(V,W) \le W^V$
\end{itemize}

\subsection*{Matrizen und lineare Abbildungen}
$A\in\Mat(m\times n, K)$.\\
$\implies f_A: K^n\to K^m$ ist $K$-linear.

\subsection*{Kern und Bild als Unterräume}
$V,W$ VR und $f:V\to W$ linear.
\begin{itemize}
	\item $U\le V \implies f(U)\le W$
	\item $U\le W \implies f^{-1}(U) \le V$
	\item $\im(f)=f(V)\le W$
	\item $\Ker(f) = \{v \mid f(v)=0\} \le V$
	\item $f$ injektiv $\iff \Ker(f)=\{0_V\}$
\end{itemize}

\subsection*{Faktorraum}
Sei $V$ VR und $U \le V$.

$V/U$ ist mit verteterweisen $(+,\cdot)$ ein VR.

\subsubsection*{Komplemente}
$V$ ist VR, $U \le V$ und $U'$ Kompl. von $U$.

$\pi_{\mid}: U' \to V/u: x \mapsto \bar x$ \\
ist Isomorphismus.

\subsection*{Homomorphiesatz}
$f: V \to W$ linear.

$\implies \bar f: V/\Ker(f) \to \Im(f): \bar x \mapsto f(x)$ \\
ist Isomorphismus.

	\section*{Basen von Vektorräumen}

\subsection*{Lineare Unabhängigkeit}
$(x_1, \dots, x_n)$ ist linear unabhängig, wenn \\
$\lambda_1\cdot x_1 + \dots + \lambda_n \cdot x_n \implies \lambda_1 = \dots = \lambda_n = 0$

\subsubsection*{Kriterien für lineare Abhängigkeit}
Wenn $0\in F$ oder $\exists x\in F$, das Linearkombination anderer Vektoren in $F$ ist, dann ist $F$ linear abhängig.

\subsection*{EZS und Basis}
$V$ ist VR, $F$ Familie aus $V$.

$F$ ist EZS von $V$ $\iff V=\Lin(F)$.

$F$ ist Basis $\iff F$ EZS und lin. unabh. 

\subsubsection*{Charakterisierung von Basen}
$B$ ist Basis bedeutet:
\begin{itemize}
	\item $B$ ist minimales EZS
	\item $B$ ist maximal lin. unabhängig
\end{itemize}

\subsection*{Austauschen in endl. VR}
\subsubsection*{Austauschlemma}
$B=(x_1,\dots,x_n)$ Basis, \\
$\displaystyle y=\sum_{i=1}^n \lambda_i x_i$ und $\lambda_j \neq 0$.

$\implies (x_1,\dots,x_{j-1},y,x_{j+1},\dots,x_n)$ Basis.

\subsubsection*{Austauschsatz von Steinitz}
$(x_1,\dots,x_n)$ Basis, $(y_1,\dots,y_r)$ lin. unabh.

$\implies x_1,\dots,x_n$ lassen sich so umnummerieren,
dass $(y_1,\dots,y_r,x_{r+1},\dots,x_n)$ Basis.

\subsubsection*{Basisergänzungssatz}
Wenn $F=(y_1,\dots,y_r)$ lin. unabhängig in $V$ (endlich),
dann kann $F$ zu einer Basis von $V$ ergänzt werden.

(Steinitz mit einer Basis und $F$)

\subsection*{Gleichmächtige Basen}
Alle Basen von $V$ sind gleichmächtig.

	\section*{Dimension von VR}

$\dim_K(V)=\begin{cases}
	n, & \text{Basis hat $n$ Elemente}\\
	\infty & \text{nicht endlich erzeugt}
\end{cases}$

\subsection*{Dimension als Invariante}
$V,W$ sind $K$-Vektorräume.

$V \cong W \iff \dim_K(V) = \dim_K(W)$

\subsubsection*{Lin. Abb. zwischen gleichdim. VR}
$V,W$ VR mit $\dim_K(V)=\dim_K(W)$, \\
$f:V\to W$ linear, dann für $f$: \\
inj. $\iff$ surj. $\iff$ bij.

\subsection*{Dimensionsformeln}
Ist $\dim_K(V)<\infty$ und $U \le V$:

$\dim_K(U) \le \dim_K(V)$ und \\
$U = V \iff \dim_K(U) = \dim_K(V)$

\subsubsection*{Für Unterräume}
$U,U' \le V$, dann:
$\dim_K(U+U')$ \\ $=\dim_K(U)+\dim_K(U')-\dim_K(U\cap U')$

\subsubsection*{Für Komplemente}
Für $U,U' \le V$ sind äquivalent:
\begin{itemize}
	\item $V=U \oplus U'$
	\item $V=U+U'$ und \\ $\dim_K V = \dim_K U + \dim_K U'$
	\item $U \cap U'$ und \\ $\dim_K V = \dim_K U + \dim_K U'$
\end{itemize}

\subsubsection*{Für Faktorräume}
$U \le V$, dann:

$\dim_K(V/U) = \dim_K(V)-\dim_K(U)$

\subsubsection*{Für lineare Abbildungen}
$f: V \to W$ linear, dann:

$\dim_K(V) = \dim_K(\Ker(f)) + \dim_K(\im(f))$

	\section*{Matrixdarstellung lin. Abb.}

\subsection*{Koordinatenvektor $M_B(x)$}
Basis $B=(b_1,\dots,b_n)$.\\
Vektor $x=(\lambda_1 b_1,\dots,\lambda_n b_n)$. \\
$\implies M_B(x) = (\lambda_1,\dots,\lambda_n)^t$

\subsection*{Matrixdarstellung $M_D^B(f)$}
Basen $B,D$ von $V$ bzw. $W$. \\
$f:V\to W$ linear.

$M_D^B(f)=(M_D(f(b_1)),\dots,M_D(f(b_n)))$

Insb.: $M_D^B: \Hom_K(V,W) \to \Mat(m \times n, K)$
ist Isomorphismus.

\subsection*{Rechnen mit Koordinaten}
\subsubsection*{Bild eines Vektors}
$f: V\to W, $ linear. $x\in V$.

$M_D(f(x))=M_D^B \circ M_B(x)$

\subsubsection*{Komposition}
$f: U\to V, g: V\to W$ linear. \\
$B,C,D$ Basen von $U,V,W$.

$M_D^B(g \circ f) = M_D^C(g) \circ M^B_C(f)$

	\section*{Basiswechsel}

\subsection*{Koordinatentransformationsmatrix $T_{B'}^B$}
$B,B'$ Basen von $V$.

$T_{B'}^B=M_{B'}^B(\id_V)=\big(M_{B'}(b_1),\dots,M_{B'}(b_n)\big)$

\subsubsection*{Invertierbarkeit}
$\left(T_{B'}^B\right)^{-1} = T_B^{B'}$

\subsection*{Anwendung}

\subsubsection*{Matrixdarstellung}
$M_{D'}^{B'}(f) = T_{D'}^D \circ M_D^B(f) \circ T_B^{B'}$

\subsubsection*{Endomorphismen}
$T = T_B^{B'}$ und $f \in\End_K(V)$.

$M_{B'}^{B'}(f) = T^{-1} \circ M_B^B(f) \circ T$

	\section*{Rang von Matrizen}

$f \in \Hom_K(V,W), A\in\Mat(m \times n, K)$.

$\rang(f) := \dim_K(\im(f))$ \\
$\rang(A) := \rang(f_A)$

\subsection*{Invertierbare Matrizen}
$A \in\Mat_n(K)$ inv.bar $\iff \rang(A) = n$

\subsection*{Rang von Matrixdarstellung}
$f\in\Hom_K(V,W)$. \\
$\rang(f) = \rang(M_D^B(f))$

\subsection*{Normalform bzgl. Äquivalenz}

\subsubsection*{Matrixdarstellung}
$f \in\Hom_K(V,W)$. \\
$\exists B,D$, so dass $M_D^B(f)$ in Normalform.

\subsubsection*{Matrix}
$A\in\Mat(m \times n, K)$. \\
$\exists S \in \Gl_m(K), T\in\Gl_n(K)$, so dass
$S \circ A \circ T$ in Normalform.

\subsection*{Zeilen- und Spaltenrang}
\subsubsection*{Transponierte und Inverse}
$A\in\Mat_n(K) \text{ inv.bar} \iff A^t \text{ inv.bar}$ \\
$\implies (A^t)^{-1} = (A^{-1})^t$

\subsubsection*{Rang der Transponierten}
$A\in\Mat(m \times n, K)$ \\
$\implies \rang(A) = \rang(A^t)$

\end{multicols*}
\end{document}
