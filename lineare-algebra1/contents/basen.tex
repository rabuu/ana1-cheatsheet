\section*{Basen von Vektorräumen}

\subsection*{Lineare Unabhängigkeit}
$(x_1, \dots, x_n)$ ist linear unabhängig, wenn \\
$\lambda_1\cdot x_1 + \dots + \lambda_n \cdot x_n \implies \lambda_1 = \dots = \lambda_n = 0$

\subsubsection*{Kriterien für lineare Abhängigkeit}
Wenn $0\in F$ oder $\exists x\in F$, das Linearkombination anderer Vektoren in $F$ ist, dann ist $F$ linear abhängig.

\subsection*{EZS und Basis}
$V$ ist VR, $F$ Familie aus $V$.

$F$ ist EZS von $V$ $\iff V=\Lin(F)$.

$F$ ist Basis $\iff F$ EZS und lin. unabh. 

\subsubsection*{Charakterisierung von Basen}
$B$ ist Basis bedeutet:
\begin{itemize}
	\item $B$ ist minimales EZS
	\item $B$ ist maximal lin. unabhängig
\end{itemize}

\subsection*{Austauschen in endl. VR}
\subsubsection*{Austauschlemma}
$B=(x_1,\dots,x_n)$ Basis, \\
$\displaystyle y=\sum_{i=1}^n \lambda_i x_i$ und $\lambda_j \neq 0$.

$\implies (x_1,\dots,x_{j-1},y,x_{j+1},\dots,x_n)$ Basis.

\subsubsection*{Austauschsatz von Steinitz}
$(x_1,\dots,x_n)$ Basis, $(y_1,\dots,y_r)$ lin. unabh.

$\implies x_1,\dots,x_n$ lassen sich so umnummerieren,
dass $(y_1,\dots,y_r,x_{r+1},\dots,x_n)$ Basis.

\subsubsection*{Basisergänzungssatz}
Wenn $F=(y_1,\dots,y_r)$ lin. unabhängig in $V$ (endlich),
dann kann $F$ zu einer Basis von $V$ ergänzt werden.

(Steinitz mit einer Basis und $F$)

\subsection*{Gleichmächtige Basen}
Alle Basen von $V$ sind gleichmächtig.
