\section*{Spektralsatz}

\subsection*{Adjungierte Abbildung $f^*$}
$f\in\End_{\K}(V)$.

$\exists_1 f^*:\langle f(x),y \rangle = \langle x, f^*(y) \rangle$

Für $(x_1,\dots,x_n)$ ONB und $y\in V$: \\
$\displaystyle f^*(y)=\sum_{i=1}^n \langle f(x_i),y \rangle \cdot x_i$

Auch: $f^{**}=f$, dh. $\langle f^*(x),y\rangle = \langle x,f(y)\rangle$

\subsubsection*{Matrixdarstellung der Adjungierten}
$B$ ist ONB und $f\in\End_{\K}(V)$.

$M_B^B(f^*)=M_B^B(f)^* = \overline{M_B^B(f)}^t$

\subsection*{Selbstadjungiert/hermitesch}
$f\in\End_{\K}(V)$, $A\in\Mat_n(\K)$.

$f$ selbstadjungiert ($\R$) bzw. \\ hermitesch ($\C$), wenn $f=f^*$.

$A$ symmetrisch ($\R$) bzw. hermitesch ($\C$), wenn $A=A^*$.

\subsubsection*{Endomorphismus und Matrix}
$f\in\End_{\K}(V)$ und $B$ eine ONB.

$f$ slbstadj./hrm. $\iff M_B^B(f)$ sym./hrm.

\subsubsection*{Eigenwerte}
$f$ selbstadjungiert.

Dann $\chi_f \in\R[t]$ und zerfällt über $\R$.
Alle Eigenwerte sind reell.

\subsection*{Spektralsatz für selbstadj. End.}
$f\in\End_{\K}(V)$ selbstadj. $\iff$ $V$ ONB aus Eigenvektoren von $f$ hat
und alle Eigenwerte reell sind.

\subsection*{Spektralsatz für sym./hrm. Matrizen}
$A\in\Mat_n(\K)$ sym./hrm.

$\exists T\in O(n)$ bzw. $U(n):$ \\
$T^{-1}\circ A \circ T=T^*\circ A \circ T=D$,
wobei $D$ Diagonalmatrix mit reellen EW ist.

$A=T\circ D \circ T^*$ ist Eigenwertzerlegung.

\subsection*{Spektralzerlegung für selbstadj. End.}
$f$ selbstadj. mit EW $\lambda_1,\dots,\lambda_r$. \\
$\pi_i$ orth. Projektion von $V$ auf $\Eig(f,\lambda_i)$.

Dann $V=\Eig(f,\lambda_i)\perp\dots\perp \Eig(f,\lambda_r)$

und $f = \lambda_1\cdot\pi_1 + \dots + \lambda_r\cdot\pi_r$.

\subsection*{Hauptachsentransformation}

\subsubsection*{Bilinearform $b$}
$b:V\times V \mapsto \R$ linear in beiden Arg.

Symmetrisch, wenn $b(x,y)=b(y,x)$.

$\rightarrow q_b: V\to\R:x\mapsto b(x,x)$ quadr. Form.

($b(x,y)=\frac{1}{2}(q_b(x+y)-q_b(x)-q_b(y))$)

\subsubsection*{Hpt.achsentransf.satz für quadr. Formen}
$A\in\Mat_n(\R)$ sym.

$\exists (x_1,\dots,x_n)$ ONB (Std.skalarprodukt) \\
mit $A\circ x_i = \lambda_i \cdot x_i$

$\displaystyle q_A(x)=\sum_{i=1}^n\lambda_i\cdot\langle x_i,x\rangle^2$

$x_i$ die Hauptachsen der quadr. Form.
