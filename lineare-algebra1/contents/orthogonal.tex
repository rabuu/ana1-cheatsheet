\section*{Orthogonal/-normal}
$V$ eukl./unit. Raum. \\
$x,y\in V,M,N\subseteq V$ und $U \le V$.

\subsection*{Orthogonal $\perp$}
$x\perp y \iff \langle x,y \rangle = 0$ \\
$M \perp N \iff \forall m\in M, n\in N: m \perp n$

\subsection*{Orthogonales Komplement $U^\perp$}
$U^\perp := \{v \in V \mid v \perp U\}$ \\
$U \le V \implies U^\perp \le V$ \\
$V = U \oplus U^\perp$

\subsection*{Orthogonal und linear unabhängig}
Eine orthogonale Familie in $V\setminus\{0\}$ ist linear unabhängig.

\subsection*{Orthonormal(-basis)}
$B=(x_i \mid i\in I)$ orthonormal, wenn
$x_i$ paarweise orthogonal und alle normiert.

$B$ Basis \rightarrow~ Orthonormalbasis (ONB).

\subsection*{Parsevalsche Gleichung}
$(b_i \mid i\in I)$ ONB.

$\displaystyle x=\sum_{i\in I}\langle b_i,x \rangle\cdot b_i$ (endlich)

\subsection*{Gram-Schmidt}
Jeder endlich-dimensionale eukl./unit. Raum besitzt eine ONB.

\subsubsection*{Orthonormalisierungsverfahren}
$M\subseteq K^n$ mit einem Skalarprodukt $\langle\cdot,\cdot\rangle$.
\begin{enumerate}
	\item Basis $(x_1,\dots,x_r)$ von $\Lin(M)$
	\item Wiederhole für $i=1,\dots,r$: \\
		$\displaystyle y_i = x_i - \sum_{j=1}^{i-1}\langle z_j,x_i \rangle\cdot z_j$ \\
		$\displaystyle z_i = \frac{1}{\|y_i\|}\cdot y_i$
	\item $(z_1,\dots,z_r)$ ist ONB von $\Lin(M)$
\end{enumerate}

\subsection*{Adjungierte Matrix}
$A\in\Mat_n(\K)$

Adjungierte $A^* := \overline{A}^t$ \\
(In $\R$ einfach $A^*=A^t$)

\subsection*{Orthogonale/unitäre Matrix}
$A$ ist orthogonal ($\R$) bzw. unitär ($\C$), wenn $A^* \circ A = \mathbbm{1}_n$

\subsubsection*{Orthogonale und unitäre Gruppe}
$O(n)$/$U(n)$ sind alle orth./unit. Matrizen.

\subsubsection*{Determinante orth./unit. Matrizen}
$A$ orth./unit. $\implies |\det(A)|=1$

\subsubsection*{Charakterisierung orth./unit. Matrizen}
Gleichbedeutend:
\begin{itemize}
	\item $A$ ist orthogonal bzw. unitär
	\item $A$ ist invertierbar mit $A^{-1}=A^*$
	\item Zeilen/Spalten von $A$ sind ONB von $\K^n$ (Standardskalarprodukt)
\end{itemize}

\subsection*{Orthogonale Summe}
$V$ eukl./unit. Raum., $U_i \le V$.

$V = U_1 \perp \dots \perp U_r$, wenn \\
$V = U_1 \oplus \dots \oplus U_r$ und alle $U_i$ paarweise orthogonal.

\subsection*{Orthogonale Projektion}
$x$ lässt sich in einem eukl./unit. Raum eindeutig als
$x=u+u'$ darstellen, wobei $u\in U$ und $u' \in U^\perp$.

$\pi_U:V\to V: x=u+u' \mapsto u$
