\section*{Eigenwerte}

\subsection*{Charakteristisches Polynom $\chi_A$}
$\chi_A := \det(t\cdot\mathbbm{1}_n - A)\in K[t]$

\subsubsection*{Spur einer Matrix}
$\Spur(A)=$ Summe der Diagonaleinträge.

\subsubsection*{Form von $\chi_A$}
$\chi_A=t^n + \alpha_{n-1}t^{n-1}+\dots+\alpha_1 t + \alpha_0 \in K[t]$
mit $\alpha_{n-1}=-\Spur(A)$ und $\alpha_0=(-1)^n\cdot\det(A)$.

\subsection*{Konjugation quadratischer Matrizen}
$A,B\in\Mat_n(K)$ konjugiert bzw. ähnlich, wenn
$\exists T\in\Gl_n(K): B=T^{-1}\circ A \circ T$

\subsubsection*{Eigenschaften}
$A,B\in\Mat_n(K)$ konjugiert: \\
$\chi_A=\chi_B,\quad \det(A)=\det(B)$ \\ und $\Spur(A)=\Spur(B)$

\subsection*{Endomorphismen}
$\dim_K(V) \ge 1$, Basis $B$ und $f\in\End_K(V)$.

$\chi_f = \chi_{M_B^B(f)},\quad \det(f)=\det(M_B^B(f))$\\
und $\Spur(f)=\Spur(M_B^B(f))$

\subsection*{Eigenwerte und -vektoren}
$\lambda \in K$, $f\in\End_K(V)$ bzw. $A\in\Mat_n(K)$.

$\exists 0\neq x\in V: f(x)=\lambda x$ \\
bzw. $\exists 0\neq x\in K^n: Ax = \lambda x$

$\implies \lambda$ ist Eigenwert, $x$ ist Eigenvektor.

\subsubsection*{Eigenräume}
$\Eig(f,\lambda) := \{y\in V \mid f(y)=\lambda y\}$ \\
$\Eig(A,\lambda) := \{y\in K^n \mid Ay=\lambda y\}$

\subsection*{Eigenwerte und charakt. Polynom}
Die Eigenwerte von $f$ bzw. $A$ sind genau die Nullstellen von $\chi_f$ bzw. $\chi_A$ in $K$.

\subsection*{Dreiecksmatrizen}
Für obere/untere Dreiecksmatr. sind die Diagonaleinträge genau die Eigenwerte.
$\chi_A=(t-a_{11})\cdot\dots\cdot(t-a_{nn})$

\subsection*{Linear unabhängige Eigenvektoren}
$x_1,\dots,x_n$ Eigenvektoren zu paarweise verschiedenen Eigenwerten.

Dann ist $(x_1,\dots,x_n)$ linear unabhängig.
