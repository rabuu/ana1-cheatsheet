\section*{Tipps \& Tricks}

\subsection*{Matrizen}

\subsubsection*{Inverse einer $2\times2$-Matrix}
$A=\mat{a & b \\ c & d}$ und $\det(A) \neq 0$.

$\displaystyle \implies A^{-1} = \frac{1}{\det(A)}\cdot\mat{d & -b \\ -c & a}$

\subsubsection*{Einfache $3\times3$-Determinante}
$A=\mat{a & 0 & b \\ 0 & c & 0 \\ d & 0 & e}$ \\
$\implies \det(A)=a\cdot c \cdot e - b \cdot c \cdot d$

\subsubsection*{Standardbasismatrizen}
$E_i^j$ hat eine 1 in der $i$-ten Zeile, $j$-ten Spalte und sonst 0.

$E_1^1 = \mat{1 & 0 \\ 0 & 0}, \quad E_1^2 = \mat{0 & 1 \\ 0 & 0}$ \\
$E_2^1 = \mat{0 & 0 \\ 1 & 0}, \quad E_2^2 = \mat{0 & 0 \\ 0 & 1}$

\subsection*{$f$-invariant}
$V$ ist VR, $f: V\to V$ linear.

$U\le V$ heißt $f$-invariant, wenn $f(U) \subseteq U$.

\subsection*{Einfach Unterräume zeigen}

\subsubsection*{Über das Bild}
Beispielaufgabe: $U$ Unterraum von $\R^2$? \\
$\displaystyle U=\left\{\mat{x+y \\ 2x-z} \mid x,y,z\in\R \right\}$

Sei $\displaystyle A=\mat{1 & 1 & 0 \\ 2 & 0 & -1} \in\Mat(2 \times 3, \R)$.

$f_A: \R^3 \to \R^2: x \mapsto Ax$ linear. \\
$\implies \im(f_A)=U \le \R^2$

\subsubsection*{Über den Kern}
Beispielaufgabe: $U'$ Unterraum von $\R^3$? \\
$\displaystyle U'=\left\{\mat{x\\y\\z}\in\R^3 \mid x=y \land 2y+3z=0\right\}$

Sei $\displaystyle A'=\mat{1 & -1 & 0 \\ 0 & 2 & 3} \in\Mat(2\times3, \R)$.

$f_{A'}: \R^3 \to \R^2: x\mapsto A'x$ linear. \\
$\implies \Ker(f_{A'})=U' \le \R^3$.

\subsection*{Orthonormalbasis bestimmen}
Rechnung vereinfacht sich oft,
wenn man zunächst eine orthogonale Basis bestimmt
und erst am Ende normiert.

\subsection*{Polynome}
\subsubsection*{Nullpolynom}
Hat ein Polynom mehr paarweise verschiedene Nullstellen als sein Grad,
ist es das Nullpolynom.

\subsection*{Permutationen}

$\displaystyle \sigma = \mat{1 & 2 & 3 & 4 & 5 \\ 3 & 4 & 5 & 2 & 1}$

\subsubsection*{Zykelschreibweise}
$\sigma = (1~3~5) \circ (2~4)$

\subsubsection*{Transpositionsschreibweise}
$\sigma = (1~3) \circ (3~5) \circ (2~4)$

Benachbarte Zahlen: \\
$\sigma = (1~2) \circ (2~3) \circ (2~1) \circ (3~4) \circ (4~5) \circ (4~3) \circ (2~3) \circ (3~4) \circ (3~2)$
