\section*{Euklidische \& unitäre Räume}
$\K\in\{\R,\C\}$

\subsection*{Komplexe Zahlen $\C$}
$z=x+iy$
\begin{itemize}
	\item $|z| = \sqrt{x^2 + y^2}$
	\item $\overline{z} = x - iy$
\end{itemize}

\subsection*{Skalarprodukt $\langle\cdot,\cdot\rangle$}
Für $\langle\cdot,\cdot\rangle: V \times V \to \K$ muss gelten:
\begin{itemize}
	\item $\langle x,\lambda y + \mu z \rangle = \lambda\cdot\langle x,y\rangle + \mu\cdot\langle x,z\rangle$
	\item $\langle x,y \rangle = \overline{\langle y,x \rangle}$
	\item $\forall x\neq 0: 0 < \langle x,x \rangle \in \R$
\end{itemize}

\subsubsection*{Standardskalarprodukt}
$\displaystyle \langle\cdot,\cdot\rangle: \K^n \times \K^n \to \K: (x,y)\mapsto \sum_{i=1}^n \overline{x_i}\cdot y_i$

\subsubsection*{Euklidischer/Unitärer Raum}
Euklidischer Raum: $\R$-VR mit Skalarpr. \\
Unitärer Raum: $\C$-VR mit Skalarprodukt.

\subsubsection*{Cauchy-Schwarzsche Ungleichung}
Im eukl./unit. Raum $V$ gilt: \\
$|\langle x,y\rangle| \le \sqrt{\langle x,x \rangle}\cdot\sqrt{\langle y,y \rangle}$

\subsection*{Norm $\|\cdot\|$}
Für $\|\cdot\|:V\to\R_{\ge 0}$ muss gelten:
\begin{itemize}
	\item $\|x\| = 0 \iff x = 0$
	\item $\|\lambda x\|=|\lambda|\cdot\|x\|$
	\item $\|x+y\| \le \|x\| + \|y\|$
\end{itemize}

\subsubsection*{Euklidische Norm}
In einem eukl./unit. Raum $V$, definiere:
$\|\cdot\|: V\to\R_{\ge0}:x\mapsto \sqrt{\langle x,x \rangle}$
