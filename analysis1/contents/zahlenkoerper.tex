\section*{Zahlenkörper}

\subsection*{Die reellen Zahlen $\mathbb{R}$}

\subsubsection*{Supremumsaxiom}
Jede nicht-leere, nach oben beschränkte Teilmenge von $\mathbb{R}$
besitzt ein $\sup$ in $\mathbb{R}$.

\subsubsection*{Archimedische Anordnung}
$\forall x,y\in\mathbb{R}, 0<x<y: \exists n\in\mathbb{N}: y<n\cdot x$ \\
$\implies \forall x:\exists n: n \le x < n+1$ \\
$\implies \forall \varepsilon>0:\exists n: 0 \le \frac{1}{n} < \varepsilon$

\subsection*{Die komplexen Zahlen $\mathbb{C}$}
\begin{itemize}
	\item $(x,y)+(u,v) = (x+u,y+v)$
	\item $(x,y)\cdot(u,v) = (xu-yv,xv-yu)$
	\item $(x,y)^{-1} = (\frac{x}{x^2+y^2}, -\frac{y}{x^2+y^2})$
\end{itemize}

Sei $z=x+iy$.
\begin{itemize}
	\item $|z|=|x+iy|=\sqrt{x^2+y^2}$
	\item $\bar{z} = x-iy$
	\item $z\cdot\bar{z}=|z|^2$
	\item $z^{-1}=\frac{1}{z}=\frac{\bar{z}}{|z|^2}$
\end{itemize}

Sei $(r,\alpha)=(|z|, \arg(z))$.
\begin{itemize}
	\item $z = r\cdot(\cos\alpha + i\cdot\sin\alpha)$
	\item Ermittlung von $\alpha$: betrachte $\frac{z}{|z|}$. \\
		($r=1$, $\arg = \alpha$)
\end{itemize}
