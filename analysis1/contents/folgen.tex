\section*{Folgen}

\subsection*{Konvergenz}
$\forall \varepsilon>0: \exists n_\varepsilon: \forall n\ge n_\varepsilon: |a_n - a| < \varepsilon$ \\
$\iff a_n \longrightarrow a$

\subsection*{Grenzwertsätze}
\begin{itemize}
	\item $a_n + b_n \longrightarrow a + b$
	\item $a_n \cdot b_n \longrightarrow a \cdot b$
	\item $|a_n| \longrightarrow |a|$
\end{itemize}

\subsection*{Einschachtelungssatz}
\begin{itemize}
	\item $a_n \le b_n \implies a \le b$
	\item $a_n \le c_n \le b_n \land a=b \implies c_n \to a$
\end{itemize}

\subsection*{Rechnen mit Nullfolgen}
$a_n$ Nullfolge, $b_n$ beschr. \\
$\implies a_n\cdot b_n$ Nullfolge

\subsection*{Monotoniekriterium}
Eine monotone und beschr. Folge konv.

\subsection*{Bolzano-Weierstraß}
Jede beschr. Flg. hat eine konv. Teilfolge.

\subsection*{Cauchy}
$\forall \varepsilon>0: \exists n_\varepsilon: \forall m>n \ge n_\varepsilon: |a_m - a_n| < \varepsilon$
$\iff$ Folge konv. (in $\mathbb{K}$)

\subsection*{Landau Notation}

\subsubsection*{Asymptotische obere Schranke}
$\frac{a_n}{b_n} \text{ beschr.} \implies a_n \in\mathcal{O}(b_n)$, $\mathcal{O}(a_n)\le\mathcal{O}(b_n)$

\subsubsection*{Scharfe asymptotische obere Schranke}
$a_n \in \mathcal{O}(b_n) \land b_n \in \mathcal{O}(a_n)$\\
$\implies a_n \in \Theta(b_n),\quad\mathcal{O}(a_n)=\mathcal{O}(b_n)$

\subsubsection*{Asymptotisch vernachlässigbar}
$\frac{a_n}{b_n} \text{ Nullfolge} \implies a_n \in o(b_n)$
